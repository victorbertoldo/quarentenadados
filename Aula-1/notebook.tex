
% Default to the notebook output style

    


% Inherit from the specified cell style.




    
\documentclass[11pt]{article}

    
    
    \usepackage[T1]{fontenc}
    % Nicer default font (+ math font) than Computer Modern for most use cases
    \usepackage{mathpazo}

    % Basic figure setup, for now with no caption control since it's done
    % automatically by Pandoc (which extracts ![](path) syntax from Markdown).
    \usepackage{graphicx}
    % We will generate all images so they have a width \maxwidth. This means
    % that they will get their normal width if they fit onto the page, but
    % are scaled down if they would overflow the margins.
    \makeatletter
    \def\maxwidth{\ifdim\Gin@nat@width>\linewidth\linewidth
    \else\Gin@nat@width\fi}
    \makeatother
    \let\Oldincludegraphics\includegraphics
    % Set max figure width to be 80% of text width, for now hardcoded.
    \renewcommand{\includegraphics}[1]{\Oldincludegraphics[width=.8\maxwidth]{#1}}
    % Ensure that by default, figures have no caption (until we provide a
    % proper Figure object with a Caption API and a way to capture that
    % in the conversion process - todo).
    \usepackage{caption}
    \DeclareCaptionLabelFormat{nolabel}{}
    \captionsetup{labelformat=nolabel}

    \usepackage{adjustbox} % Used to constrain images to a maximum size 
    \usepackage{xcolor} % Allow colors to be defined
    \usepackage{enumerate} % Needed for markdown enumerations to work
    \usepackage{geometry} % Used to adjust the document margins
    \usepackage{amsmath} % Equations
    \usepackage{amssymb} % Equations
    \usepackage{textcomp} % defines textquotesingle
    % Hack from http://tex.stackexchange.com/a/47451/13684:
    \AtBeginDocument{%
        \def\PYZsq{\textquotesingle}% Upright quotes in Pygmentized code
    }
    \usepackage{upquote} % Upright quotes for verbatim code
    \usepackage{eurosym} % defines \euro
    \usepackage[mathletters]{ucs} % Extended unicode (utf-8) support
    \usepackage[utf8x]{inputenc} % Allow utf-8 characters in the tex document
    \usepackage{fancyvrb} % verbatim replacement that allows latex
    \usepackage{grffile} % extends the file name processing of package graphics 
                         % to support a larger range 
    % The hyperref package gives us a pdf with properly built
    % internal navigation ('pdf bookmarks' for the table of contents,
    % internal cross-reference links, web links for URLs, etc.)
    \usepackage{hyperref}
    \usepackage{longtable} % longtable support required by pandoc >1.10
    \usepackage{booktabs}  % table support for pandoc > 1.12.2
    \usepackage[inline]{enumitem} % IRkernel/repr support (it uses the enumerate* environment)
    \usepackage[normalem]{ulem} % ulem is needed to support strikethroughs (\sout)
                                % normalem makes italics be italics, not underlines
    

    
    
    % Colors for the hyperref package
    \definecolor{urlcolor}{rgb}{0,.145,.698}
    \definecolor{linkcolor}{rgb}{.71,0.21,0.01}
    \definecolor{citecolor}{rgb}{.12,.54,.11}

    % ANSI colors
    \definecolor{ansi-black}{HTML}{3E424D}
    \definecolor{ansi-black-intense}{HTML}{282C36}
    \definecolor{ansi-red}{HTML}{E75C58}
    \definecolor{ansi-red-intense}{HTML}{B22B31}
    \definecolor{ansi-green}{HTML}{00A250}
    \definecolor{ansi-green-intense}{HTML}{007427}
    \definecolor{ansi-yellow}{HTML}{DDB62B}
    \definecolor{ansi-yellow-intense}{HTML}{B27D12}
    \definecolor{ansi-blue}{HTML}{208FFB}
    \definecolor{ansi-blue-intense}{HTML}{0065CA}
    \definecolor{ansi-magenta}{HTML}{D160C4}
    \definecolor{ansi-magenta-intense}{HTML}{A03196}
    \definecolor{ansi-cyan}{HTML}{60C6C8}
    \definecolor{ansi-cyan-intense}{HTML}{258F8F}
    \definecolor{ansi-white}{HTML}{C5C1B4}
    \definecolor{ansi-white-intense}{HTML}{A1A6B2}

    % commands and environments needed by pandoc snippets
    % extracted from the output of `pandoc -s`
    \providecommand{\tightlist}{%
      \setlength{\itemsep}{0pt}\setlength{\parskip}{0pt}}
    \DefineVerbatimEnvironment{Highlighting}{Verbatim}{commandchars=\\\{\}}
    % Add ',fontsize=\small' for more characters per line
    \newenvironment{Shaded}{}{}
    \newcommand{\KeywordTok}[1]{\textcolor[rgb]{0.00,0.44,0.13}{\textbf{{#1}}}}
    \newcommand{\DataTypeTok}[1]{\textcolor[rgb]{0.56,0.13,0.00}{{#1}}}
    \newcommand{\DecValTok}[1]{\textcolor[rgb]{0.25,0.63,0.44}{{#1}}}
    \newcommand{\BaseNTok}[1]{\textcolor[rgb]{0.25,0.63,0.44}{{#1}}}
    \newcommand{\FloatTok}[1]{\textcolor[rgb]{0.25,0.63,0.44}{{#1}}}
    \newcommand{\CharTok}[1]{\textcolor[rgb]{0.25,0.44,0.63}{{#1}}}
    \newcommand{\StringTok}[1]{\textcolor[rgb]{0.25,0.44,0.63}{{#1}}}
    \newcommand{\CommentTok}[1]{\textcolor[rgb]{0.38,0.63,0.69}{\textit{{#1}}}}
    \newcommand{\OtherTok}[1]{\textcolor[rgb]{0.00,0.44,0.13}{{#1}}}
    \newcommand{\AlertTok}[1]{\textcolor[rgb]{1.00,0.00,0.00}{\textbf{{#1}}}}
    \newcommand{\FunctionTok}[1]{\textcolor[rgb]{0.02,0.16,0.49}{{#1}}}
    \newcommand{\RegionMarkerTok}[1]{{#1}}
    \newcommand{\ErrorTok}[1]{\textcolor[rgb]{1.00,0.00,0.00}{\textbf{{#1}}}}
    \newcommand{\NormalTok}[1]{{#1}}
    
    % Additional commands for more recent versions of Pandoc
    \newcommand{\ConstantTok}[1]{\textcolor[rgb]{0.53,0.00,0.00}{{#1}}}
    \newcommand{\SpecialCharTok}[1]{\textcolor[rgb]{0.25,0.44,0.63}{{#1}}}
    \newcommand{\VerbatimStringTok}[1]{\textcolor[rgb]{0.25,0.44,0.63}{{#1}}}
    \newcommand{\SpecialStringTok}[1]{\textcolor[rgb]{0.73,0.40,0.53}{{#1}}}
    \newcommand{\ImportTok}[1]{{#1}}
    \newcommand{\DocumentationTok}[1]{\textcolor[rgb]{0.73,0.13,0.13}{\textit{{#1}}}}
    \newcommand{\AnnotationTok}[1]{\textcolor[rgb]{0.38,0.63,0.69}{\textbf{\textit{{#1}}}}}
    \newcommand{\CommentVarTok}[1]{\textcolor[rgb]{0.38,0.63,0.69}{\textbf{\textit{{#1}}}}}
    \newcommand{\VariableTok}[1]{\textcolor[rgb]{0.10,0.09,0.49}{{#1}}}
    \newcommand{\ControlFlowTok}[1]{\textcolor[rgb]{0.00,0.44,0.13}{\textbf{{#1}}}}
    \newcommand{\OperatorTok}[1]{\textcolor[rgb]{0.40,0.40,0.40}{{#1}}}
    \newcommand{\BuiltInTok}[1]{{#1}}
    \newcommand{\ExtensionTok}[1]{{#1}}
    \newcommand{\PreprocessorTok}[1]{\textcolor[rgb]{0.74,0.48,0.00}{{#1}}}
    \newcommand{\AttributeTok}[1]{\textcolor[rgb]{0.49,0.56,0.16}{{#1}}}
    \newcommand{\InformationTok}[1]{\textcolor[rgb]{0.38,0.63,0.69}{\textbf{\textit{{#1}}}}}
    \newcommand{\WarningTok}[1]{\textcolor[rgb]{0.38,0.63,0.69}{\textbf{\textit{{#1}}}}}
    
    
    % Define a nice break command that doesn't care if a line doesn't already
    % exist.
    \def\br{\hspace*{\fill} \\* }
    % Math Jax compatability definitions
    \def\gt{>}
    \def\lt{<}
    % Document parameters
    \title{QuarentenaDados Filmes}
    
    
    

    % Pygments definitions
    
\makeatletter
\def\PY@reset{\let\PY@it=\relax \let\PY@bf=\relax%
    \let\PY@ul=\relax \let\PY@tc=\relax%
    \let\PY@bc=\relax \let\PY@ff=\relax}
\def\PY@tok#1{\csname PY@tok@#1\endcsname}
\def\PY@toks#1+{\ifx\relax#1\empty\else%
    \PY@tok{#1}\expandafter\PY@toks\fi}
\def\PY@do#1{\PY@bc{\PY@tc{\PY@ul{%
    \PY@it{\PY@bf{\PY@ff{#1}}}}}}}
\def\PY#1#2{\PY@reset\PY@toks#1+\relax+\PY@do{#2}}

\expandafter\def\csname PY@tok@w\endcsname{\def\PY@tc##1{\textcolor[rgb]{0.73,0.73,0.73}{##1}}}
\expandafter\def\csname PY@tok@c\endcsname{\let\PY@it=\textit\def\PY@tc##1{\textcolor[rgb]{0.25,0.50,0.50}{##1}}}
\expandafter\def\csname PY@tok@cp\endcsname{\def\PY@tc##1{\textcolor[rgb]{0.74,0.48,0.00}{##1}}}
\expandafter\def\csname PY@tok@k\endcsname{\let\PY@bf=\textbf\def\PY@tc##1{\textcolor[rgb]{0.00,0.50,0.00}{##1}}}
\expandafter\def\csname PY@tok@kp\endcsname{\def\PY@tc##1{\textcolor[rgb]{0.00,0.50,0.00}{##1}}}
\expandafter\def\csname PY@tok@kt\endcsname{\def\PY@tc##1{\textcolor[rgb]{0.69,0.00,0.25}{##1}}}
\expandafter\def\csname PY@tok@o\endcsname{\def\PY@tc##1{\textcolor[rgb]{0.40,0.40,0.40}{##1}}}
\expandafter\def\csname PY@tok@ow\endcsname{\let\PY@bf=\textbf\def\PY@tc##1{\textcolor[rgb]{0.67,0.13,1.00}{##1}}}
\expandafter\def\csname PY@tok@nb\endcsname{\def\PY@tc##1{\textcolor[rgb]{0.00,0.50,0.00}{##1}}}
\expandafter\def\csname PY@tok@nf\endcsname{\def\PY@tc##1{\textcolor[rgb]{0.00,0.00,1.00}{##1}}}
\expandafter\def\csname PY@tok@nc\endcsname{\let\PY@bf=\textbf\def\PY@tc##1{\textcolor[rgb]{0.00,0.00,1.00}{##1}}}
\expandafter\def\csname PY@tok@nn\endcsname{\let\PY@bf=\textbf\def\PY@tc##1{\textcolor[rgb]{0.00,0.00,1.00}{##1}}}
\expandafter\def\csname PY@tok@ne\endcsname{\let\PY@bf=\textbf\def\PY@tc##1{\textcolor[rgb]{0.82,0.25,0.23}{##1}}}
\expandafter\def\csname PY@tok@nv\endcsname{\def\PY@tc##1{\textcolor[rgb]{0.10,0.09,0.49}{##1}}}
\expandafter\def\csname PY@tok@no\endcsname{\def\PY@tc##1{\textcolor[rgb]{0.53,0.00,0.00}{##1}}}
\expandafter\def\csname PY@tok@nl\endcsname{\def\PY@tc##1{\textcolor[rgb]{0.63,0.63,0.00}{##1}}}
\expandafter\def\csname PY@tok@ni\endcsname{\let\PY@bf=\textbf\def\PY@tc##1{\textcolor[rgb]{0.60,0.60,0.60}{##1}}}
\expandafter\def\csname PY@tok@na\endcsname{\def\PY@tc##1{\textcolor[rgb]{0.49,0.56,0.16}{##1}}}
\expandafter\def\csname PY@tok@nt\endcsname{\let\PY@bf=\textbf\def\PY@tc##1{\textcolor[rgb]{0.00,0.50,0.00}{##1}}}
\expandafter\def\csname PY@tok@nd\endcsname{\def\PY@tc##1{\textcolor[rgb]{0.67,0.13,1.00}{##1}}}
\expandafter\def\csname PY@tok@s\endcsname{\def\PY@tc##1{\textcolor[rgb]{0.73,0.13,0.13}{##1}}}
\expandafter\def\csname PY@tok@sd\endcsname{\let\PY@it=\textit\def\PY@tc##1{\textcolor[rgb]{0.73,0.13,0.13}{##1}}}
\expandafter\def\csname PY@tok@si\endcsname{\let\PY@bf=\textbf\def\PY@tc##1{\textcolor[rgb]{0.73,0.40,0.53}{##1}}}
\expandafter\def\csname PY@tok@se\endcsname{\let\PY@bf=\textbf\def\PY@tc##1{\textcolor[rgb]{0.73,0.40,0.13}{##1}}}
\expandafter\def\csname PY@tok@sr\endcsname{\def\PY@tc##1{\textcolor[rgb]{0.73,0.40,0.53}{##1}}}
\expandafter\def\csname PY@tok@ss\endcsname{\def\PY@tc##1{\textcolor[rgb]{0.10,0.09,0.49}{##1}}}
\expandafter\def\csname PY@tok@sx\endcsname{\def\PY@tc##1{\textcolor[rgb]{0.00,0.50,0.00}{##1}}}
\expandafter\def\csname PY@tok@m\endcsname{\def\PY@tc##1{\textcolor[rgb]{0.40,0.40,0.40}{##1}}}
\expandafter\def\csname PY@tok@gh\endcsname{\let\PY@bf=\textbf\def\PY@tc##1{\textcolor[rgb]{0.00,0.00,0.50}{##1}}}
\expandafter\def\csname PY@tok@gu\endcsname{\let\PY@bf=\textbf\def\PY@tc##1{\textcolor[rgb]{0.50,0.00,0.50}{##1}}}
\expandafter\def\csname PY@tok@gd\endcsname{\def\PY@tc##1{\textcolor[rgb]{0.63,0.00,0.00}{##1}}}
\expandafter\def\csname PY@tok@gi\endcsname{\def\PY@tc##1{\textcolor[rgb]{0.00,0.63,0.00}{##1}}}
\expandafter\def\csname PY@tok@gr\endcsname{\def\PY@tc##1{\textcolor[rgb]{1.00,0.00,0.00}{##1}}}
\expandafter\def\csname PY@tok@ge\endcsname{\let\PY@it=\textit}
\expandafter\def\csname PY@tok@gs\endcsname{\let\PY@bf=\textbf}
\expandafter\def\csname PY@tok@gp\endcsname{\let\PY@bf=\textbf\def\PY@tc##1{\textcolor[rgb]{0.00,0.00,0.50}{##1}}}
\expandafter\def\csname PY@tok@go\endcsname{\def\PY@tc##1{\textcolor[rgb]{0.53,0.53,0.53}{##1}}}
\expandafter\def\csname PY@tok@gt\endcsname{\def\PY@tc##1{\textcolor[rgb]{0.00,0.27,0.87}{##1}}}
\expandafter\def\csname PY@tok@err\endcsname{\def\PY@bc##1{\setlength{\fboxsep}{0pt}\fcolorbox[rgb]{1.00,0.00,0.00}{1,1,1}{\strut ##1}}}
\expandafter\def\csname PY@tok@kc\endcsname{\let\PY@bf=\textbf\def\PY@tc##1{\textcolor[rgb]{0.00,0.50,0.00}{##1}}}
\expandafter\def\csname PY@tok@kd\endcsname{\let\PY@bf=\textbf\def\PY@tc##1{\textcolor[rgb]{0.00,0.50,0.00}{##1}}}
\expandafter\def\csname PY@tok@kn\endcsname{\let\PY@bf=\textbf\def\PY@tc##1{\textcolor[rgb]{0.00,0.50,0.00}{##1}}}
\expandafter\def\csname PY@tok@kr\endcsname{\let\PY@bf=\textbf\def\PY@tc##1{\textcolor[rgb]{0.00,0.50,0.00}{##1}}}
\expandafter\def\csname PY@tok@bp\endcsname{\def\PY@tc##1{\textcolor[rgb]{0.00,0.50,0.00}{##1}}}
\expandafter\def\csname PY@tok@fm\endcsname{\def\PY@tc##1{\textcolor[rgb]{0.00,0.00,1.00}{##1}}}
\expandafter\def\csname PY@tok@vc\endcsname{\def\PY@tc##1{\textcolor[rgb]{0.10,0.09,0.49}{##1}}}
\expandafter\def\csname PY@tok@vg\endcsname{\def\PY@tc##1{\textcolor[rgb]{0.10,0.09,0.49}{##1}}}
\expandafter\def\csname PY@tok@vi\endcsname{\def\PY@tc##1{\textcolor[rgb]{0.10,0.09,0.49}{##1}}}
\expandafter\def\csname PY@tok@vm\endcsname{\def\PY@tc##1{\textcolor[rgb]{0.10,0.09,0.49}{##1}}}
\expandafter\def\csname PY@tok@sa\endcsname{\def\PY@tc##1{\textcolor[rgb]{0.73,0.13,0.13}{##1}}}
\expandafter\def\csname PY@tok@sb\endcsname{\def\PY@tc##1{\textcolor[rgb]{0.73,0.13,0.13}{##1}}}
\expandafter\def\csname PY@tok@sc\endcsname{\def\PY@tc##1{\textcolor[rgb]{0.73,0.13,0.13}{##1}}}
\expandafter\def\csname PY@tok@dl\endcsname{\def\PY@tc##1{\textcolor[rgb]{0.73,0.13,0.13}{##1}}}
\expandafter\def\csname PY@tok@s2\endcsname{\def\PY@tc##1{\textcolor[rgb]{0.73,0.13,0.13}{##1}}}
\expandafter\def\csname PY@tok@sh\endcsname{\def\PY@tc##1{\textcolor[rgb]{0.73,0.13,0.13}{##1}}}
\expandafter\def\csname PY@tok@s1\endcsname{\def\PY@tc##1{\textcolor[rgb]{0.73,0.13,0.13}{##1}}}
\expandafter\def\csname PY@tok@mb\endcsname{\def\PY@tc##1{\textcolor[rgb]{0.40,0.40,0.40}{##1}}}
\expandafter\def\csname PY@tok@mf\endcsname{\def\PY@tc##1{\textcolor[rgb]{0.40,0.40,0.40}{##1}}}
\expandafter\def\csname PY@tok@mh\endcsname{\def\PY@tc##1{\textcolor[rgb]{0.40,0.40,0.40}{##1}}}
\expandafter\def\csname PY@tok@mi\endcsname{\def\PY@tc##1{\textcolor[rgb]{0.40,0.40,0.40}{##1}}}
\expandafter\def\csname PY@tok@il\endcsname{\def\PY@tc##1{\textcolor[rgb]{0.40,0.40,0.40}{##1}}}
\expandafter\def\csname PY@tok@mo\endcsname{\def\PY@tc##1{\textcolor[rgb]{0.40,0.40,0.40}{##1}}}
\expandafter\def\csname PY@tok@ch\endcsname{\let\PY@it=\textit\def\PY@tc##1{\textcolor[rgb]{0.25,0.50,0.50}{##1}}}
\expandafter\def\csname PY@tok@cm\endcsname{\let\PY@it=\textit\def\PY@tc##1{\textcolor[rgb]{0.25,0.50,0.50}{##1}}}
\expandafter\def\csname PY@tok@cpf\endcsname{\let\PY@it=\textit\def\PY@tc##1{\textcolor[rgb]{0.25,0.50,0.50}{##1}}}
\expandafter\def\csname PY@tok@c1\endcsname{\let\PY@it=\textit\def\PY@tc##1{\textcolor[rgb]{0.25,0.50,0.50}{##1}}}
\expandafter\def\csname PY@tok@cs\endcsname{\let\PY@it=\textit\def\PY@tc##1{\textcolor[rgb]{0.25,0.50,0.50}{##1}}}

\def\PYZbs{\char`\\}
\def\PYZus{\char`\_}
\def\PYZob{\char`\{}
\def\PYZcb{\char`\}}
\def\PYZca{\char`\^}
\def\PYZam{\char`\&}
\def\PYZlt{\char`\<}
\def\PYZgt{\char`\>}
\def\PYZsh{\char`\#}
\def\PYZpc{\char`\%}
\def\PYZdl{\char`\$}
\def\PYZhy{\char`\-}
\def\PYZsq{\char`\'}
\def\PYZdq{\char`\"}
\def\PYZti{\char`\~}
% for compatibility with earlier versions
\def\PYZat{@}
\def\PYZlb{[}
\def\PYZrb{]}
\makeatother


    % Exact colors from NB
    \definecolor{incolor}{rgb}{0.0, 0.0, 0.5}
    \definecolor{outcolor}{rgb}{0.545, 0.0, 0.0}



    
    % Prevent overflowing lines due to hard-to-break entities
    \sloppy 
    % Setup hyperref package
    \hypersetup{
      breaklinks=true,  % so long urls are correctly broken across lines
      colorlinks=true,
      urlcolor=urlcolor,
      linkcolor=linkcolor,
      citecolor=citecolor,
      }
    % Slightly bigger margins than the latex defaults
    
    \geometry{verbose,tmargin=1in,bmargin=1in,lmargin=1in,rmargin=1in}
    
    

    \begin{document}
    
    
    \maketitle
    
    

    
    \hypertarget{introduuxe7uxe3o-em-analise-de-dados-em-python}{%
\section{Introdução em analise de dados em
python}\label{introduuxe7uxe3o-em-analise-de-dados-em-python}}

    \hypertarget{apenas-executando-alguns-cuxf3digos}{%
\subsubsection{Apenas executando alguns
códigos}\label{apenas-executando-alguns-cuxf3digos}}

    \begin{Verbatim}[commandchars=\\\{\}]
{\color{incolor}In [{\color{incolor}1}]:} \PY{n+nb}{print}\PY{p}{(}\PY{l+s+s2}{\PYZdq{}}\PY{l+s+s2}{Victor  Bertoldo}\PY{l+s+s2}{\PYZdq{}}\PY{p}{)}
\end{Verbatim}


    \begin{Verbatim}[commandchars=\\\{\}]
Victor  Bertoldo

    \end{Verbatim}

    \begin{Verbatim}[commandchars=\\\{\}]
{\color{incolor}In [{\color{incolor}2}]:} \PY{n}{nome\PYZus{}do\PYZus{}filme} \PY{o}{=} \PY{l+s+s1}{\PYZsq{}}\PY{l+s+s1}{Totoro, o filme}\PY{l+s+s1}{\PYZsq{}}
\end{Verbatim}


    \begin{Verbatim}[commandchars=\\\{\}]
{\color{incolor}In [{\color{incolor}3}]:} \PY{n+nb}{print}\PY{p}{(}\PY{n}{nome\PYZus{}do\PYZus{}filme}\PY{p}{)}
\end{Verbatim}


    \begin{Verbatim}[commandchars=\\\{\}]
Totoro, o filme

    \end{Verbatim}

    \begin{Verbatim}[commandchars=\\\{\}]
{\color{incolor}In [{\color{incolor}4}]:} \PY{n}{nome\PYZus{}do\PYZus{}filme}
\end{Verbatim}


\begin{Verbatim}[commandchars=\\\{\}]
{\color{outcolor}Out[{\color{outcolor}4}]:} 'Totoro, o filme'
\end{Verbatim}
            
    \hypertarget{importando-a-base-de-dados-de-filmes-usando-pandas}{%
\subsection{Importando a base de dados de filmes usando
pandas}\label{importando-a-base-de-dados-de-filmes-usando-pandas}}

    \begin{Verbatim}[commandchars=\\\{\}]
{\color{incolor}In [{\color{incolor}5}]:} \PY{k+kn}{import} \PY{n+nn}{pandas} \PY{k}{as} \PY{n+nn}{pd}
        
        \PY{n}{uri} \PY{o}{=} \PY{l+s+s1}{\PYZsq{}}\PY{l+s+s1}{https://raw.githubusercontent.com/alura\PYZhy{}cursos/introducao\PYZhy{}a\PYZhy{}data\PYZhy{}science/master/aula0/ml\PYZhy{}latest\PYZhy{}small/movies.csv}\PY{l+s+s1}{\PYZsq{}}
        \PY{n}{filmes} \PY{o}{=} \PY{n}{pd}\PY{o}{.}\PY{n}{read\PYZus{}csv}\PY{p}{(}\PY{n}{uri}\PY{p}{)}
        
        \PY{c+c1}{\PYZsh{} filmes é um dataframe}
        \PY{n}{filmes}\PY{o}{.}\PY{n}{head}\PY{p}{(}\PY{p}{)}
\end{Verbatim}


\begin{Verbatim}[commandchars=\\\{\}]
{\color{outcolor}Out[{\color{outcolor}5}]:}    movieId                               title  \textbackslash{}
        0        1                    Toy Story (1995)   
        1        2                      Jumanji (1995)   
        2        3             Grumpier Old Men (1995)   
        3        4            Waiting to Exhale (1995)   
        4        5  Father of the Bride Part II (1995)   
        
                                                genres  
        0  Adventure|Animation|Children|Comedy|Fantasy  
        1                   Adventure|Children|Fantasy  
        2                               Comedy|Romance  
        3                         Comedy|Drama|Romance  
        4                                       Comedy  
\end{Verbatim}
            
    ** Caso queira ver a documentação referente ao tipo de objeto
(variavel), basta colocar um `?' após a variável **

    \begin{Verbatim}[commandchars=\\\{\}]
{\color{incolor}In [{\color{incolor}6}]:} \PY{c+c1}{\PYZsh{}filmes?}
\end{Verbatim}


    \hypertarget{alterando-o-nome-das-colunas}{%
\subsubsection{Alterando o nome das
colunas}\label{alterando-o-nome-das-colunas}}

    \begin{Verbatim}[commandchars=\\\{\}]
{\color{incolor}In [{\color{incolor}7}]:} \PY{n}{filmes}\PY{o}{.}\PY{n}{columns} \PY{o}{=} \PY{p}{[}\PY{l+s+s1}{\PYZsq{}}\PY{l+s+s1}{filmeId}\PY{l+s+s1}{\PYZsq{}}\PY{p}{,} \PY{l+s+s1}{\PYZsq{}}\PY{l+s+s1}{filme}\PY{l+s+s1}{\PYZsq{}}\PY{p}{,} \PY{l+s+s1}{\PYZsq{}}\PY{l+s+s1}{genero}\PY{l+s+s1}{\PYZsq{}} \PY{p}{]}
\end{Verbatim}


    \hypertarget{importando-base-de-dados-de-avaliauxe7uxf5es}{%
\subsection{Importando base de dados de
avaliações}\label{importando-base-de-dados-de-avaliauxe7uxf5es}}

    \begin{Verbatim}[commandchars=\\\{\}]
{\color{incolor}In [{\color{incolor}8}]:} \PY{n}{uri} \PY{o}{=} \PY{l+s+s1}{\PYZsq{}}\PY{l+s+s1}{https://github.com/alura\PYZhy{}cursos/introducao\PYZhy{}a\PYZhy{}data\PYZhy{}science/blob/master/aula0/ml\PYZhy{}latest\PYZhy{}small/ratings.csv?raw=true}\PY{l+s+s1}{\PYZsq{}}
        
        \PY{n}{avaliacoes} \PY{o}{=} \PY{n}{pd}\PY{o}{.}\PY{n}{read\PYZus{}csv}\PY{p}{(}\PY{n}{uri}\PY{p}{)}
        \PY{n}{avaliacoes}\PY{o}{.}\PY{n}{head}\PY{p}{(}\PY{p}{)}
\end{Verbatim}


\begin{Verbatim}[commandchars=\\\{\}]
{\color{outcolor}Out[{\color{outcolor}8}]:}    userId  movieId  rating  timestamp
        0       1        1     4.0  964982703
        1       1        3     4.0  964981247
        2       1        6     4.0  964982224
        3       1       47     5.0  964983815
        4       1       50     5.0  964982931
\end{Verbatim}
            
    \hypertarget{para-visualizar-a-qtd-de-linhas-e-colunas-do-dataframe}{%
\subsubsection{Para visualizar a Qtd de linhas e colunas do
dataframe}\label{para-visualizar-a-qtd-de-linhas-e-colunas-do-dataframe}}

    \begin{Verbatim}[commandchars=\\\{\}]
{\color{incolor}In [{\color{incolor}9}]:} \PY{n}{avaliacoes}\PY{o}{.}\PY{n}{shape} 
\end{Verbatim}


\begin{Verbatim}[commandchars=\\\{\}]
{\color{outcolor}Out[{\color{outcolor}9}]:} (100836, 4)
\end{Verbatim}
            
    \hypertarget{para-visualizar-apenas-a-qtd-de-linhas}{%
\subsubsection{Para visualizar apenas a qtd de
linhas}\label{para-visualizar-apenas-a-qtd-de-linhas}}

    \begin{Verbatim}[commandchars=\\\{\}]
{\color{incolor}In [{\color{incolor}10}]:} \PY{n+nb}{len}\PY{p}{(}\PY{n}{avaliacoes}\PY{p}{)}
\end{Verbatim}


\begin{Verbatim}[commandchars=\\\{\}]
{\color{outcolor}Out[{\color{outcolor}10}]:} 100836
\end{Verbatim}
            
    \hypertarget{sobreescrevendo-o-nome-das-colunas}{%
\subsubsection{Sobreescrevendo o nome das
colunas}\label{sobreescrevendo-o-nome-das-colunas}}

    \begin{Verbatim}[commandchars=\\\{\}]
{\color{incolor}In [{\color{incolor}11}]:} \PY{n}{avaliacoes}\PY{o}{.}\PY{n}{columns}
         
         \PY{n}{avaliacoes}\PY{o}{.}\PY{n}{columns} \PY{o}{=} \PY{p}{[}\PY{l+s+s1}{\PYZsq{}}\PY{l+s+s1}{usuarioId}\PY{l+s+s1}{\PYZsq{}}\PY{p}{,} \PY{l+s+s1}{\PYZsq{}}\PY{l+s+s1}{filmeId}\PY{l+s+s1}{\PYZsq{}}\PY{p}{,} \PY{l+s+s1}{\PYZsq{}}\PY{l+s+s1}{nota}\PY{l+s+s1}{\PYZsq{}}\PY{p}{,} \PY{l+s+s1}{\PYZsq{}}\PY{l+s+s1}{momento}\PY{l+s+s1}{\PYZsq{}}\PY{p}{]}
         \PY{n}{avaliacoes}\PY{o}{.}\PY{n}{head}\PY{p}{(}\PY{p}{)}
\end{Verbatim}


\begin{Verbatim}[commandchars=\\\{\}]
{\color{outcolor}Out[{\color{outcolor}11}]:}    usuarioId  filmeId  nota    momento
         0          1        1   4.0  964982703
         1          1        3   4.0  964981247
         2          1        6   4.0  964982224
         3          1       47   5.0  964983815
         4          1       50   5.0  964982931
\end{Verbatim}
            
    \hypertarget{fazendo-consultas-no-dataframe}{%
\subsection{Fazendo consultas no
dataframe}\label{fazendo-consultas-no-dataframe}}

\hypertarget{trazendo-as-avaliauxe7uxf5es-apenas-do-filmeid-igual-uxe0-1.-mostrando-apenas-os-10-primeiros}{%
\subsubsection{Trazendo as avaliações apenas do filmeId igual à 1.
Mostrando apenas os 10
primeiros}\label{trazendo-as-avaliauxe7uxf5es-apenas-do-filmeid-igual-uxe0-1.-mostrando-apenas-os-10-primeiros}}

\begin{quote}
Para trabalharmos com esta consulta, vamos armazenar estes dados em uma
variável
\end{quote}

    \begin{Verbatim}[commandchars=\\\{\}]
{\color{incolor}In [{\color{incolor}12}]:} \PY{n}{avaliacoes\PYZus{}filmeId\PYZus{}1} \PY{o}{=} \PY{n}{avaliacoes}\PY{o}{.}\PY{n}{query}\PY{p}{(}\PY{l+s+s1}{\PYZsq{}}\PY{l+s+s1}{filmeId==1}\PY{l+s+s1}{\PYZsq{}}\PY{p}{)}
         
         \PY{n}{avaliacoes\PYZus{}filmeId\PYZus{}1}\PY{o}{.}\PY{n}{head}\PY{p}{(}\PY{n}{n}\PY{o}{=}\PY{l+m+mi}{10}\PY{p}{)}
\end{Verbatim}


\begin{Verbatim}[commandchars=\\\{\}]
{\color{outcolor}Out[{\color{outcolor}12}]:}       usuarioId  filmeId  nota     momento
         0             1        1   4.0   964982703
         516           5        1   4.0   847434962
         874           7        1   4.5  1106635946
         1434         15        1   2.5  1510577970
         1667         17        1   4.5  1305696483
         1772         18        1   3.5  1455209816
         2274         19        1   4.0   965705637
         3219         21        1   3.5  1407618878
         4059         27        1   3.0   962685262
         4879         31        1   5.0   850466616
\end{Verbatim}
            
    \hypertarget{o-describe-mostra-algumas-estatisticas-do-dataframe}{%
\subsection{O describe mostra algumas estatisticas do
dataframe}\label{o-describe-mostra-algumas-estatisticas-do-dataframe}}

\hypertarget{mostrando-as-estatisticas-do-dataframe-completo}{%
\subsubsection{Mostrando as estatisticas do dataframe
completo:}\label{mostrando-as-estatisticas-do-dataframe-completo}}

    \begin{Verbatim}[commandchars=\\\{\}]
{\color{incolor}In [{\color{incolor}13}]:} \PY{n}{avaliacoes}\PY{o}{.}\PY{n}{describe}\PY{p}{(}\PY{p}{)}
\end{Verbatim}


\begin{Verbatim}[commandchars=\\\{\}]
{\color{outcolor}Out[{\color{outcolor}13}]:}            usuarioId        filmeId           nota       momento
         count  100836.000000  100836.000000  100836.000000  1.008360e+05
         mean      326.127564   19435.295718       3.501557  1.205946e+09
         std       182.618491   35530.987199       1.042529  2.162610e+08
         min         1.000000       1.000000       0.500000  8.281246e+08
         25\%       177.000000    1199.000000       3.000000  1.019124e+09
         50\%       325.000000    2991.000000       3.500000  1.186087e+09
         75\%       477.000000    8122.000000       4.000000  1.435994e+09
         max       610.000000  193609.000000       5.000000  1.537799e+09
\end{Verbatim}
            
    \hypertarget{visuxe3o-preview-de-apenas-uma-coluna}{%
\paragraph{Visão (preview) de apenas uma
coluna}\label{visuxe3o-preview-de-apenas-uma-coluna}}

    \begin{Verbatim}[commandchars=\\\{\}]
{\color{incolor}In [{\color{incolor}14}]:} \PY{n}{avaliacoes}\PY{p}{[}\PY{l+s+s1}{\PYZsq{}}\PY{l+s+s1}{nota}\PY{l+s+s1}{\PYZsq{}}\PY{p}{]}\PY{o}{.}\PY{n}{head}\PY{p}{(}\PY{p}{)}
\end{Verbatim}


\begin{Verbatim}[commandchars=\\\{\}]
{\color{outcolor}Out[{\color{outcolor}14}]:} 0    4.0
         1    4.0
         2    4.0
         3    5.0
         4    5.0
         Name: nota, dtype: float64
\end{Verbatim}
            
    \hypertarget{estatuxedsticas-apenas-do-filme-de-id-igual-a-1}{%
\subsection{Estatísticas apenas do filme de Id igual a
1}\label{estatuxedsticas-apenas-do-filme-de-id-igual-a-1}}

    \begin{Verbatim}[commandchars=\\\{\}]
{\color{incolor}In [{\color{incolor}15}]:} \PY{n}{avaliacoes\PYZus{}filmeId\PYZus{}1}\PY{o}{.}\PY{n}{describe}\PY{p}{(}\PY{p}{)}
\end{Verbatim}


\begin{Verbatim}[commandchars=\\\{\}]
{\color{outcolor}Out[{\color{outcolor}15}]:}         usuarioId  filmeId        nota       momento
         count  215.000000    215.0  215.000000  2.150000e+02
         mean   306.530233      1.0    3.920930  1.129835e+09
         std    180.419754      0.0    0.834859  2.393163e+08
         min      1.000000      1.0    0.500000  8.293223e+08
         25\%    155.500000      1.0    3.500000  8.779224e+08
         50\%    290.000000      1.0    4.000000  1.106855e+09
         75\%    468.500000      1.0    4.500000  1.348523e+09
         max    610.000000      1.0    5.000000  1.535710e+09
\end{Verbatim}
            
    \hypertarget{definidndo-o-id-do-filme-que-queremos-analisar}{%
\paragraph{Definidndo o id do filme que queremos
analisar}\label{definidndo-o-id-do-filme-que-queremos-analisar}}

    \begin{Verbatim}[commandchars=\\\{\}]
{\color{incolor}In [{\color{incolor}16}]:} \PY{n+nb}{id} \PY{o}{=} \PY{l+m+mi}{1}
\end{Verbatim}


    \hypertarget{armazenando-o-nome-do-filme-em-uma-variuxe1vel}{%
\paragraph{Armazenando o nome do filme em uma
variável}\label{armazenando-o-nome-do-filme-em-uma-variuxe1vel}}

    \begin{Verbatim}[commandchars=\\\{\}]
{\color{incolor}In [{\color{incolor}17}]:} \PY{n}{nome\PYZus{}do\PYZus{}filme} \PY{o}{=} \PY{n}{filmes}\PY{o}{.}\PY{n}{query}\PY{p}{(}\PY{l+s+s2}{\PYZdq{}}\PY{l+s+s2}{filmeId==@id}\PY{l+s+s2}{\PYZdq{}}\PY{p}{)}\PY{p}{[}\PY{l+s+s1}{\PYZsq{}}\PY{l+s+s1}{filme}\PY{l+s+s1}{\PYZsq{}}\PY{p}{]}
         \PY{n}{nome\PYZus{}do\PYZus{}filme}
\end{Verbatim}


\begin{Verbatim}[commandchars=\\\{\}]
{\color{outcolor}Out[{\color{outcolor}17}]:} 0    Toy Story (1995)
         Name: filme, dtype: object
\end{Verbatim}
            
    \hypertarget{trazendo-muxe9tricas-especuxedficas-de-campo}{%
\subsection{Trazendo métricas específicas de
campo}\label{trazendo-muxe9tricas-especuxedficas-de-campo}}

\hypertarget{buscando-apenas-o-nome-do-filme}{%
\paragraph{Buscando apenas o nome do
Filme}\label{buscando-apenas-o-nome-do-filme}}

    \begin{Verbatim}[commandchars=\\\{\}]
{\color{incolor}In [{\color{incolor}18}]:} \PY{n}{f1} \PY{o}{=} \PY{n+nb}{str}\PY{p}{(}\PY{n}{nome\PYZus{}do\PYZus{}filme}\PY{p}{)}\PY{o}{.}\PY{n}{strip}\PY{p}{(}\PY{p}{)}\PY{o}{.}\PY{n}{split}\PY{p}{(}\PY{p}{)}\PY{p}{[}\PY{l+m+mi}{1}\PY{p}{:}\PY{l+m+mi}{3}\PY{p}{]}
         
         \PY{n}{f1} \PY{o}{=} \PY{n}{f1}\PY{p}{[}\PY{l+m+mi}{0}\PY{p}{]} \PY{o}{+} \PY{l+s+s1}{\PYZsq{}}\PY{l+s+s1}{ }\PY{l+s+s1}{\PYZsq{}} \PY{o}{+} \PY{n}{f1}\PY{p}{[}\PY{l+m+mi}{1}\PY{p}{]}
         \PY{n+nb}{print}\PY{p}{(}\PY{n}{f1}\PY{p}{)}
\end{Verbatim}


    \begin{Verbatim}[commandchars=\\\{\}]
Toy Story

    \end{Verbatim}

    \hypertarget{apresentando-a-muxe9dia-das-notas-apenas-do-filme-id-igual-uxe0-1}{%
\subsection{Apresentando a Média das notas apenas do filme id igual à
1}\label{apresentando-a-muxe9dia-das-notas-apenas-do-filme-id-igual-uxe0-1}}

    \begin{Verbatim}[commandchars=\\\{\}]
{\color{incolor}In [{\color{incolor}19}]:} \PY{n}{mediaf1} \PY{o}{=} \PY{n}{avaliacoes\PYZus{}filmeId\PYZus{}1}\PY{p}{[}\PY{l+s+s1}{\PYZsq{}}\PY{l+s+s1}{nota}\PY{l+s+s1}{\PYZsq{}}\PY{p}{]}\PY{o}{.}\PY{n}{mean}\PY{p}{(}\PY{p}{)}
         
         \PY{n+nb}{print}\PY{p}{(}\PY{l+s+s2}{\PYZdq{}}\PY{l+s+s2}{A média do Filme }\PY{l+s+si}{\PYZob{}\PYZcb{}}\PY{l+s+s2}{, é }\PY{l+s+si}{\PYZob{}:.2f\PYZcb{}}\PY{l+s+s2}{.}\PY{l+s+s2}{\PYZdq{}}\PY{o}{.}\PY{n}{format}\PY{p}{(}\PY{n}{f1}\PY{p}{,}\PY{n}{mediaf1}\PY{p}{)}\PY{p}{)}
\end{Verbatim}


    \begin{Verbatim}[commandchars=\\\{\}]
A média do Filme Toy Story, é 3.92.

    \end{Verbatim}

    \hypertarget{cruzando-os-dados-dos-2-dataframes-para-mostrar-tabela-com-o-nome-do-filme-e-a-sua-muxe9dia-de-avaliauxe7uxf5es}{%
\section{Cruzando os dados dos 2 dataframes para mostrar tabela com o
nome do filme e a sua média de
avaliações}\label{cruzando-os-dados-dos-2-dataframes-para-mostrar-tabela-com-o-nome-do-filme-e-a-sua-muxe9dia-de-avaliauxe7uxf5es}}

    ** Agrupando dos dados da tabela de avaliações e trazendo a média por
filme **

    \begin{Verbatim}[commandchars=\\\{\}]
{\color{incolor}In [{\color{incolor}20}]:} \PY{n}{notas\PYZus{}medias\PYZus{}por\PYZus{}filme} \PY{o}{=} \PY{n}{avaliacoes}\PY{o}{.}\PY{n}{groupby}\PY{p}{(}\PY{l+s+s2}{\PYZdq{}}\PY{l+s+s2}{filmeId}\PY{l+s+s2}{\PYZdq{}}\PY{p}{)}\PY{p}{[}\PY{l+s+s2}{\PYZdq{}}\PY{l+s+s2}{nota}\PY{l+s+s2}{\PYZdq{}}\PY{p}{]}\PY{o}{.}\PY{n}{mean}\PY{p}{(}\PY{p}{)}
         \PY{n}{notas\PYZus{}medias\PYZus{}por\PYZus{}filme}\PY{o}{.}\PY{n}{head}\PY{p}{(}\PY{p}{)}
\end{Verbatim}


\begin{Verbatim}[commandchars=\\\{\}]
{\color{outcolor}Out[{\color{outcolor}20}]:} filmeId
         1    3.920930
         2    3.431818
         3    3.259615
         4    2.357143
         5    3.071429
         Name: nota, dtype: float64
\end{Verbatim}
            
    \hypertarget{usando-o-join}{%
\paragraph{Usando o join}\label{usando-o-join}}

    \begin{quote}
Se o numero de linhas do dataframe `avaliacoes' fosse examente igual ao
numero de linhas da series `notas\_medias\_por\_filme' e os dois
estivessem na mesma ordem, eu poderia apenas colar a series no dataframe
da seguinte maneira:
\end{quote}

\begin{verbatim}
filmes["nota_media"] = notas_medias_por_filme
filmes.head()
\end{verbatim}

    \begin{Verbatim}[commandchars=\\\{\}]
{\color{incolor}In [{\color{incolor}21}]:} \PY{n}{filmes\PYZus{}com\PYZus{}media} \PY{o}{=} \PY{n}{filmes}\PY{o}{.}\PY{n}{join}\PY{p}{(}\PY{n}{notas\PYZus{}medias\PYZus{}por\PYZus{}filme}\PY{p}{,} \PY{n}{on}\PY{o}{=}\PY{l+s+s2}{\PYZdq{}}\PY{l+s+s2}{filmeId}\PY{l+s+s2}{\PYZdq{}}\PY{p}{)}
         
         \PY{n}{filmes\PYZus{}com\PYZus{}media}\PY{o}{.}\PY{n}{head}\PY{p}{(}\PY{p}{)}
\end{Verbatim}


\begin{Verbatim}[commandchars=\\\{\}]
{\color{outcolor}Out[{\color{outcolor}21}]:}    filmeId                               filme  \textbackslash{}
         0        1                    Toy Story (1995)   
         1        2                      Jumanji (1995)   
         2        3             Grumpier Old Men (1995)   
         3        4            Waiting to Exhale (1995)   
         4        5  Father of the Bride Part II (1995)   
         
                                                 genero      nota  
         0  Adventure|Animation|Children|Comedy|Fantasy  3.920930  
         1                   Adventure|Children|Fantasy  3.431818  
         2                               Comedy|Romance  3.259615  
         3                         Comedy|Drama|Romance  2.357143  
         4                                       Comedy  3.071429  
\end{Verbatim}
            
    \hypertarget{ordenando-o-dataframe-por-nota-de-forma-decrescente}{%
\paragraph{Ordenando o dataframe por nota de forma
decrescente}\label{ordenando-o-dataframe-por-nota-de-forma-decrescente}}

    \begin{Verbatim}[commandchars=\\\{\}]
{\color{incolor}In [{\color{incolor}22}]:} \PY{n}{filmes\PYZus{}com\PYZus{}media}\PY{o}{.}\PY{n}{sort\PYZus{}values}\PY{p}{(}\PY{l+s+s1}{\PYZsq{}}\PY{l+s+s1}{nota}\PY{l+s+s1}{\PYZsq{}}\PY{p}{,} \PY{n}{ascending}\PY{o}{=}\PY{k+kc}{False}\PY{p}{)}\PY{o}{.}\PY{n}{head}\PY{p}{(}\PY{l+m+mi}{15}\PY{p}{)}
\end{Verbatim}


\begin{Verbatim}[commandchars=\\\{\}]
{\color{outcolor}Out[{\color{outcolor}22}]:}       filmeId                                              filme  \textbackslash{}
         7656    88448              Paper Birds (Pájaros de papel) (2010)   
         8107   100556                         Act of Killing, The (2012)   
         9083   143031                                    Jump In! (2007)   
         9094   143511                                       Human (2015)   
         9096   143559                                L.A. Slasher (2015)   
         4251     6201                                   Lady Jane (1986)   
         8154   102217                     Bill Hicks: Revelations (1993)   
         8148   102084                       Justice League: Doom (2012)    
         4246     6192          Open Hearts (Elsker dig for evigt) (2002)   
         9122   145994                             Formula of Love (1984)   
         8115   100906                                Maniac Cop 2 (1990)   
         9129   146662             Dragons: Gift of the Night Fury (2011)   
         8074    99636                            English Vinglish (2012)   
         5785    31522  Marriage of Maria Braun, The (Ehe der Maria Br{\ldots}   
         9131   146684                      Cosmic Scrat-tastrophe (2015)   
         
                                   genero  nota  
         7656                Comedy|Drama   5.0  
         8107                 Documentary   5.0  
         9083        Comedy|Drama|Romance   5.0  
         9094                 Documentary   5.0  
         9096        Comedy|Crime|Fantasy   5.0  
         4251               Drama|Romance   5.0  
         8154                      Comedy   5.0  
         8148    Action|Animation|Fantasy   5.0  
         4246                     Romance   5.0  
         9122                      Comedy   5.0  
         8115      Action|Horror|Thriller   5.0  
         9129  Adventure|Animation|Comedy   5.0  
         8074                Comedy|Drama   5.0  
         5785                       Drama   5.0  
         9131   Animation|Children|Comedy   5.0  
\end{Verbatim}
            
    \hypertarget{iniciando-visualizauxe7uxe3o-de-dados}{%
\section{Iniciando visualização de
dados}\label{iniciando-visualizauxe7uxe3o-de-dados}}

\hypertarget{hipuxf3tese-1}{%
\subsubsection{Hipótese 1:}\label{hipuxf3tese-1}}

\begin{itemize}
\tightlist
\item
  Será que os filmes com média igual a nota máxima receberam o mesmo
  quantitativo de votos que os filmes com média entre 3 e 4?
\item
  Qual será a representatividade dos votos de acordo com o filme?
\item
  Um filme com média 5 pode ter recebido apenas 1 voto e em
  contrapartida um filme com média 4 pode ter recebido 100 votos, o que
  não significa que o filme com média 5 é ``melhor'' que outros filmes
  com média inferior.
\end{itemize}

\begin{quote}
Para validar a hipótese serão analisados os filmes 1, 2 e 102084
\end{quote}

    \hypertarget{muxe9dia-dos-filmes-escolhidos}{%
\subsubsection{Média dos filmes
escolhidos:}\label{muxe9dia-dos-filmes-escolhidos}}

    \begin{Verbatim}[commandchars=\\\{\}]
{\color{incolor}In [{\color{incolor}23}]:} \PY{n}{filmes\PYZus{}com\PYZus{}media}\PY{o}{.}\PY{n}{query}\PY{p}{(}\PY{l+s+s2}{\PYZdq{}}\PY{l+s+s2}{filmeId in [1,2, 102084]}\PY{l+s+s2}{\PYZdq{}}\PY{p}{)}
\end{Verbatim}


\begin{Verbatim}[commandchars=\\\{\}]
{\color{outcolor}Out[{\color{outcolor}23}]:}       filmeId                         filme  \textbackslash{}
         0           1              Toy Story (1995)   
         1           2                Jumanji (1995)   
         8148   102084  Justice League: Doom (2012)    
         
                                                    genero      nota  
         0     Adventure|Animation|Children|Comedy|Fantasy  3.920930  
         1                      Adventure|Children|Fantasy  3.431818  
         8148                     Action|Animation|Fantasy  5.000000  
\end{Verbatim}
            
    \hypertarget{importando-biblioteca-para-refinar-as-visualizauxe7uxf5es}{%
\paragraph{Importando biblioteca para refinar as
visualizações}\label{importando-biblioteca-para-refinar-as-visualizauxe7uxf5es}}

    \begin{Verbatim}[commandchars=\\\{\}]
{\color{incolor}In [{\color{incolor}24}]:} \PY{k+kn}{import} \PY{n+nn}{matplotlib}\PY{n+nn}{.}\PY{n+nn}{pyplot} \PY{k}{as} \PY{n+nn}{plt}
\end{Verbatim}


    \hypertarget{distribuiuxe7uxe3o-dos-dados-de-voto-do-filme-1}{%
\subsubsection{Distribuição dos dados de voto do filme
1}\label{distribuiuxe7uxe3o-dos-dados-de-voto-do-filme-1}}

    \begin{Verbatim}[commandchars=\\\{\}]
{\color{incolor}In [{\color{incolor}25}]:} \PY{n}{avaliacoes}\PY{o}{.}\PY{n}{query}\PY{p}{(}\PY{l+s+s2}{\PYZdq{}}\PY{l+s+s2}{filmeId==1}\PY{l+s+s2}{\PYZdq{}}\PY{p}{)}\PY{p}{[}\PY{l+s+s2}{\PYZdq{}}\PY{l+s+s2}{nota}\PY{l+s+s2}{\PYZdq{}}\PY{p}{]}\PY{o}{.}\PY{n}{plot}\PY{p}{(}\PY{n}{kind}\PY{o}{=}\PY{l+s+s1}{\PYZsq{}}\PY{l+s+s1}{hist}\PY{l+s+s1}{\PYZsq{}}\PY{p}{,} \PY{n}{title}\PY{o}{=}\PY{l+s+s1}{\PYZsq{}}\PY{l+s+s1}{Avaliações do filme Toy Story}\PY{l+s+s1}{\PYZsq{}}\PY{p}{)}
         \PY{n}{plt}\PY{o}{.}\PY{n}{show}\PY{p}{(}\PY{p}{)}
\end{Verbatim}


    \begin{center}
    \adjustimage{max size={0.9\linewidth}{0.9\paperheight}}{output_50_0.png}
    \end{center}
    { \hspace*{\fill} \\}
    
    \begin{quote}
Neste filme houve uma quantidade considerável de votos com a nota
máxima.
\end{quote}

    \hypertarget{distribuiuxe7uxe3o-dos-dados-de-voto-do-filme-2}{%
\subsubsection{Distribuição dos dados de voto do filme
2}\label{distribuiuxe7uxe3o-dos-dados-de-voto-do-filme-2}}

    \begin{Verbatim}[commandchars=\\\{\}]
{\color{incolor}In [{\color{incolor}26}]:} \PY{n}{avaliacoes}\PY{o}{.}\PY{n}{query}\PY{p}{(}\PY{l+s+s2}{\PYZdq{}}\PY{l+s+s2}{filmeId==2}\PY{l+s+s2}{\PYZdq{}}\PY{p}{)}\PY{p}{[}\PY{l+s+s2}{\PYZdq{}}\PY{l+s+s2}{nota}\PY{l+s+s2}{\PYZdq{}}\PY{p}{]}\PY{o}{.}\PY{n}{plot}\PY{p}{(}\PY{n}{kind}\PY{o}{=}\PY{l+s+s1}{\PYZsq{}}\PY{l+s+s1}{hist}\PY{l+s+s1}{\PYZsq{}}\PY{p}{,} \PY{n}{title}\PY{o}{=}\PY{l+s+s1}{\PYZsq{}}\PY{l+s+s1}{Avaliações do filme Jumanji}\PY{l+s+s1}{\PYZsq{}}\PY{p}{)}
         \PY{n}{plt}\PY{o}{.}\PY{n}{show}\PY{p}{(}\PY{p}{)}
\end{Verbatim}


    \begin{center}
    \adjustimage{max size={0.9\linewidth}{0.9\paperheight}}{output_53_0.png}
    \end{center}
    { \hspace*{\fill} \\}
    
    \begin{quote}
Este filme recebeu menos votos com notas máximas e mais votos entre 3 e
4.
\end{quote}

    \hypertarget{distribuiuxe7uxe3o-dos-dados-de-voto-do-filme-102084}{%
\subsubsection{Distribuição dos dados de voto do filme
102084}\label{distribuiuxe7uxe3o-dos-dados-de-voto-do-filme-102084}}

    \begin{Verbatim}[commandchars=\\\{\}]
{\color{incolor}In [{\color{incolor}27}]:} \PY{n}{avaliacoes}\PY{o}{.}\PY{n}{query}\PY{p}{(}\PY{l+s+s2}{\PYZdq{}}\PY{l+s+s2}{filmeId==102084}\PY{l+s+s2}{\PYZdq{}}\PY{p}{)}\PY{p}{[}\PY{l+s+s2}{\PYZdq{}}\PY{l+s+s2}{nota}\PY{l+s+s2}{\PYZdq{}}\PY{p}{]}\PY{o}{.}\PY{n}{plot}\PY{p}{(}\PY{n}{kind}\PY{o}{=}\PY{l+s+s1}{\PYZsq{}}\PY{l+s+s1}{hist}\PY{l+s+s1}{\PYZsq{}}\PY{p}{,} \PY{n}{title}\PY{o}{=}\PY{l+s+s1}{\PYZsq{}}\PY{l+s+s1}{Justice League: Doom}\PY{l+s+s1}{\PYZsq{}}\PY{p}{)}
         \PY{n}{plt}\PY{o}{.}\PY{n}{show}\PY{p}{(}\PY{p}{)}
\end{Verbatim}


    \begin{center}
    \adjustimage{max size={0.9\linewidth}{0.9\paperheight}}{output_56_0.png}
    \end{center}
    { \hspace*{\fill} \\}
    
    \begin{quote}
Este filme obteve nota máxima, porém de apenas 1 voto.
\end{quote}

    \hypertarget{diante-da-amostra-analisada-acima-a-hipuxf3tese-se-confirmou}{%
\subsection{Diante da amostra analisada acima, a hipótese se
confirmou!}\label{diante-da-amostra-analisada-acima-a-hipuxf3tese-se-confirmou}}

\hypertarget{avaliando-a-distribuiuxe7uxe3o-dos-dados-fica-claro-que-o-filme-com-nota-muxe1xima-nuxe3o-obteve-um-nuxfamero-de-votos-representativos-o-suficiente-para-determinar-que-os-filmes-melhores-avaliados-nota-muxe1xima-nuxe3o-necessariamente-suxe3o-filmes-melhores-que-filmes-com-muxe9dia-entre-3-e-4.}{%
\paragraph{Avaliando a distribuição dos dados, fica claro que o filme
com nota máxima não obteve um número de votos representativos o
suficiente para determinar que os filmes melhores avaliados (nota
máxima) não necessariamente são filmes melhores que filmes com média
entre 3 e
4.}\label{avaliando-a-distribuiuxe7uxe3o-dos-dados-fica-claro-que-o-filme-com-nota-muxe1xima-nuxe3o-obteve-um-nuxfamero-de-votos-representativos-o-suficiente-para-determinar-que-os-filmes-melhores-avaliados-nota-muxe1xima-nuxe3o-necessariamente-suxe3o-filmes-melhores-que-filmes-com-muxe9dia-entre-3-e-4.}}

    \hypertarget{desafio-1}{%
\section{Desafio 1:}\label{desafio-1}}

Encontrar os filmes que não tiveram nenhuma avaliação

    \hypertarget{visualizando-a-coluna-nota}{%
\subsubsection{Visualizando a coluna
nota:}\label{visualizando-a-coluna-nota}}

    \begin{Verbatim}[commandchars=\\\{\}]
{\color{incolor}In [{\color{incolor}28}]:} \PY{n}{avaliacoes}\PY{p}{[}\PY{l+s+s1}{\PYZsq{}}\PY{l+s+s1}{nota}\PY{l+s+s1}{\PYZsq{}}\PY{p}{]}\PY{o}{.}\PY{n}{head}\PY{p}{(}\PY{l+m+mi}{10}\PY{p}{)}
\end{Verbatim}


\begin{Verbatim}[commandchars=\\\{\}]
{\color{outcolor}Out[{\color{outcolor}28}]:} 0    4.0
         1    4.0
         2    4.0
         3    5.0
         4    5.0
         5    3.0
         6    5.0
         7    4.0
         8    5.0
         9    5.0
         Name: nota, dtype: float64
\end{Verbatim}
            
    \hypertarget{quantas-linhas-existem-na-coluna-nota}{%
\subsubsection{Quantas linhas existem na coluna
nota?}\label{quantas-linhas-existem-na-coluna-nota}}

    \begin{Verbatim}[commandchars=\\\{\}]
{\color{incolor}In [{\color{incolor}29}]:} \PY{n}{avaliacoes}\PY{p}{[}\PY{l+s+s1}{\PYZsq{}}\PY{l+s+s1}{nota}\PY{l+s+s1}{\PYZsq{}}\PY{p}{]}\PY{o}{.}\PY{n}{count}\PY{p}{(}\PY{p}{)}
\end{Verbatim}


\begin{Verbatim}[commandchars=\\\{\}]
{\color{outcolor}Out[{\color{outcolor}29}]:} 100836
\end{Verbatim}
            
    \hypertarget{transformando-a-coluna-nota-em-uma-series-indexada-pelo-id-do-filme}{%
\subsubsection{Transformando a coluna Nota em uma Series indexada pelo
id do
filme}\label{transformando-a-coluna-nota-em-uma-series-indexada-pelo-id-do-filme}}

    \begin{Verbatim}[commandchars=\\\{\}]
{\color{incolor}In [{\color{incolor}30}]:} \PY{n}{series\PYZus{}nota} \PY{o}{=} \PY{n}{pd}\PY{o}{.}\PY{n}{Series}\PY{p}{(}\PY{n}{avaliacoes}\PY{p}{[}\PY{l+s+s1}{\PYZsq{}}\PY{l+s+s1}{nota}\PY{l+s+s1}{\PYZsq{}}\PY{p}{]}\PY{p}{,} \PY{n}{index}\PY{o}{=}\PY{n}{avaliacoes}\PY{p}{[}\PY{l+s+s1}{\PYZsq{}}\PY{l+s+s1}{filmeId}\PY{l+s+s1}{\PYZsq{}}\PY{p}{]}\PY{p}{)}
\end{Verbatim}


    \hypertarget{retornando-o-total-de-avaliauxe7uxf5es-nulas}{%
\subsubsection{Retornando o total de avaliações
nulas}\label{retornando-o-total-de-avaliauxe7uxf5es-nulas}}

    \begin{Verbatim}[commandchars=\\\{\}]
{\color{incolor}In [{\color{incolor}33}]:} \PY{n}{nulos} \PY{o}{=} \PY{n}{series\PYZus{}nota}\PY{o}{.}\PY{n}{isna}\PY{p}{(}\PY{p}{)}\PY{o}{.}\PY{n}{where}\PY{p}{(}\PY{n}{series\PYZus{}nota} \PY{o}{==} \PY{k+kc}{True}\PY{p}{)}
         \PY{n}{nulos}\PY{o}{.}\PY{n}{count}\PY{p}{(}\PY{p}{)}
\end{Verbatim}


\begin{Verbatim}[commandchars=\\\{\}]
{\color{outcolor}Out[{\color{outcolor}33}]:} 3284
\end{Verbatim}
            
    \hypertarget{isso-nuxe3o-quer-dizer-que-este-valor-de-nulos-uxe9-total-de-filmes-sem-avaliauxe7uxf5es}{%
\subsubsection{Isso não quer dizer que este valor de nulos é total de
filmes sem
avaliações}\label{isso-nuxe3o-quer-dizer-que-este-valor-de-nulos-uxe9-total-de-filmes-sem-avaliauxe7uxf5es}}

** Para isso, vamos primeiro buscar o dataframe de filmes com média **

    \hypertarget{buscar-apenas-filmes-sem-avaliauxe7uxf5es}{%
\subsubsection{Buscar apenas filmes sem
avaliações}\label{buscar-apenas-filmes-sem-avaliauxe7uxf5es}}

\begin{quote}
A ideia aqui é conseguir criar uma lista apenas com o index de filmes
sem avaliações, após obter a lista de index, usá-la como parâmetro no
loc para trazer do dataframe `filmes\_com\_media' os filmes sem
avaliações.
\end{quote}

    \begin{Verbatim}[commandchars=\\\{\}]
{\color{incolor}In [{\color{incolor}35}]:} \PY{n}{lista\PYZus{}nulos} \PY{o}{=} \PY{n}{pd}\PY{o}{.}\PY{n}{isnull}\PY{p}{(}\PY{n}{filmes\PYZus{}com\PYZus{}media}\PY{p}{)}
         \PY{n}{indice} \PY{o}{=} \PY{n}{lista\PYZus{}nulos}\PY{o}{.}\PY{n}{query}\PY{p}{(}\PY{l+s+s2}{\PYZdq{}}\PY{l+s+s2}{nota==True}\PY{l+s+s2}{\PYZdq{}}\PY{p}{)}\PY{o}{.}\PY{n}{index}
         \PY{n}{filmes\PYZus{}com\PYZus{}avaliacoes\PYZus{}nulas} \PY{o}{=} \PY{n}{filmes\PYZus{}com\PYZus{}media}\PY{o}{.}\PY{n}{loc}\PY{p}{[}\PY{n}{indice}\PY{p}{]}
         \PY{n}{filmes\PYZus{}com\PYZus{}avaliacoes\PYZus{}nulas}
\end{Verbatim}


\begin{Verbatim}[commandchars=\\\{\}]
{\color{outcolor}Out[{\color{outcolor}35}]:}       filmeId                                         filme  \textbackslash{}
         816      1076                         Innocents, The (1961)   
         2211     2939                                Niagara (1953)   
         2499     3338                        For All Mankind (1989)   
         2587     3456  Color of Paradise, The (Rang-e khoda) (1999)   
         3118     4194                I Know Where I'm Going! (1945)   
         4037     5721                            Chosen, The (1981)   
         4506     6668   Road Home, The (Wo de fu qin mu qin) (1999)   
         4598     6849                                Scrooge (1970)   
         4704     7020                                  Proof (1991)   
         5020     7792                     Parallax View, The (1974)   
         5293     8765                      This Gun for Hire (1942)   
         5421    25855                  Roaring Twenties, The (1939)   
         5452    26085                   Mutiny on the Bounty (1962)   
         5749    30892            In the Realms of the Unreal (2004)   
         5824    32160                      Twentieth Century (1934)   
         5837    32371                     Call Northside 777 (1948)   
         5957    34482                  Browning Version, The (1951)   
         7565    85565                            Chalet Girl (2011)   
         
                                 genero  nota  
         816      Drama|Horror|Thriller   NaN  
         2211            Drama|Thriller   NaN  
         2499               Documentary   NaN  
         2587                     Drama   NaN  
         3118         Drama|Romance|War   NaN  
         4037                     Drama   NaN  
         4506             Drama|Romance   NaN  
         4598     Drama|Fantasy|Musical   NaN  
         4704      Comedy|Drama|Romance   NaN  
         5020                  Thriller   NaN  
         5293  Crime|Film-Noir|Thriller   NaN  
         5421      Crime|Drama|Thriller   NaN  
         5452   Adventure|Drama|Romance   NaN  
         5749     Animation|Documentary   NaN  
         5824                    Comedy   NaN  
         5837     Crime|Drama|Film-Noir   NaN  
         5957                     Drama   NaN  
         7565            Comedy|Romance   NaN  
\end{Verbatim}
            
    \hypertarget{outra-forma-mais-simples-de-trazer-estes-dados-uxe9-da-maneira-mostrada-abaixo}{%
\subsubsection{Outra forma mais simples de trazer estes dados é da
maneira mostrada
abaixo:}\label{outra-forma-mais-simples-de-trazer-estes-dados-uxe9-da-maneira-mostrada-abaixo}}

    \begin{Verbatim}[commandchars=\\\{\}]
{\color{incolor}In [{\color{incolor}36}]:} \PY{n}{lista\PYZus{}nulo} \PY{o}{=} \PY{n}{filmes\PYZus{}com\PYZus{}media}\PY{p}{[}\PY{l+s+s1}{\PYZsq{}}\PY{l+s+s1}{nota}\PY{l+s+s1}{\PYZsq{}}\PY{p}{]}\PY{o}{.}\PY{n}{isnull}\PY{p}{(}\PY{p}{)}
         \PY{n}{filmes\PYZus{}com\PYZus{}media}\PY{p}{[}\PY{n}{lista\PYZus{}nulo}\PY{p}{]}\PY{o}{.}\PY{n}{head}\PY{p}{(}\PY{l+m+mi}{3}\PY{p}{)}
\end{Verbatim}


\begin{Verbatim}[commandchars=\\\{\}]
{\color{outcolor}Out[{\color{outcolor}36}]:}       filmeId                   filme                 genero  nota
         816      1076   Innocents, The (1961)  Drama|Horror|Thriller   NaN
         2211     2939          Niagara (1953)         Drama|Thriller   NaN
         2499     3338  For All Mankind (1989)            Documentary   NaN
\end{Verbatim}
            
    \begin{Verbatim}[commandchars=\\\{\}]
{\color{incolor}In [{\color{incolor}37}]:} \PY{n}{qtd\PYZus{}nulo} \PY{o}{=} \PY{n}{filmes\PYZus{}com\PYZus{}avaliacoes\PYZus{}nulas}\PY{p}{[}\PY{l+s+s1}{\PYZsq{}}\PY{l+s+s1}{filmeId}\PY{l+s+s1}{\PYZsq{}}\PY{p}{]}\PY{o}{.}\PY{n}{count}\PY{p}{(}\PY{p}{)}
         \PY{n+nb}{print}\PY{p}{(}\PY{l+s+s2}{\PYZdq{}}\PY{l+s+s2}{No total houveram }\PY{l+s+si}{\PYZob{}\PYZcb{}}\PY{l+s+s2}{ filmes sem nenhuma avalição.}\PY{l+s+s2}{\PYZdq{}}\PY{o}{.}\PY{n}{format}\PY{p}{(}\PY{n}{qtd\PYZus{}nulo}\PY{p}{)}\PY{p}{)}
\end{Verbatim}


    \begin{Verbatim}[commandchars=\\\{\}]
No total houveram 18 filmes sem nenhuma avalição.

    \end{Verbatim}

    \hypertarget{desafio-2}{%
\section{Desafio 2:}\label{desafio-2}}

Mudar o nome da coluna nota para média após o join.

    \begin{Verbatim}[commandchars=\\\{\}]
{\color{incolor}In [{\color{incolor}38}]:} \PY{n}{filmes\PYZus{}com\PYZus{}media}\PY{o}{.}\PY{n}{rename}\PY{p}{(}\PY{n}{columns}\PY{o}{=}\PY{p}{\PYZob{}}\PY{l+s+s1}{\PYZsq{}}\PY{l+s+s1}{nota}\PY{l+s+s1}{\PYZsq{}}\PY{p}{:}\PY{l+s+s1}{\PYZsq{}}\PY{l+s+s1}{media\PYZus{}notas}\PY{l+s+s1}{\PYZsq{}}\PY{p}{\PYZcb{}}\PY{p}{,} \PY{n}{inplace}\PY{o}{=}\PY{k+kc}{True}\PY{p}{)}
         \PY{n}{filmes\PYZus{}com\PYZus{}media}\PY{o}{.}\PY{n}{head}\PY{p}{(}\PY{p}{)}
\end{Verbatim}


\begin{Verbatim}[commandchars=\\\{\}]
{\color{outcolor}Out[{\color{outcolor}38}]:}    filmeId                               filme  \textbackslash{}
         0        1                    Toy Story (1995)   
         1        2                      Jumanji (1995)   
         2        3             Grumpier Old Men (1995)   
         3        4            Waiting to Exhale (1995)   
         4        5  Father of the Bride Part II (1995)   
         
                                                 genero  media\_notas  
         0  Adventure|Animation|Children|Comedy|Fantasy     3.920930  
         1                   Adventure|Children|Fantasy     3.431818  
         2                               Comedy|Romance     3.259615  
         3                         Comedy|Drama|Romance     2.357143  
         4                                       Comedy     3.071429  
\end{Verbatim}
            
    \hypertarget{desafio-3}{%
\section{Desafio 3}\label{desafio-3}}

Colocar o número de avaliações por filme, isto é, não só a media, mas o
total de votos por filme

    \begin{Verbatim}[commandchars=\\\{\}]
{\color{incolor}In [{\color{incolor}39}]:} \PY{n}{notas\PYZus{}aggr\PYZus{}qtd} \PY{o}{=} \PY{n}{avaliacoes}\PY{o}{.}\PY{n}{groupby}\PY{p}{(}\PY{l+s+s2}{\PYZdq{}}\PY{l+s+s2}{filmeId}\PY{l+s+s2}{\PYZdq{}}\PY{p}{)}\PY{p}{[}\PY{l+s+s1}{\PYZsq{}}\PY{l+s+s1}{nota}\PY{l+s+s1}{\PYZsq{}}\PY{p}{]}\PY{o}{.}\PY{n}{count}\PY{p}{(}\PY{p}{)}
         \PY{n}{notas\PYZus{}aggr\PYZus{}qtd}\PY{o}{.}\PY{n}{head}\PY{p}{(}\PY{p}{)}
\end{Verbatim}


\begin{Verbatim}[commandchars=\\\{\}]
{\color{outcolor}Out[{\color{outcolor}39}]:} filmeId
         1    215
         2    110
         3     52
         4      7
         5     49
         Name: nota, dtype: int64
\end{Verbatim}
            
    \hypertarget{dataset-de-filmes-com-a-qtd-de-avaliauxe7uxf5es}{%
\subsubsection{Dataset de filmes com a qtd de
avaliações}\label{dataset-de-filmes-com-a-qtd-de-avaliauxe7uxf5es}}

    \begin{Verbatim}[commandchars=\\\{\}]
{\color{incolor}In [{\color{incolor}40}]:} \PY{n}{filmes\PYZus{}com\PYZus{}qtd} \PY{o}{=} \PY{n}{filmes\PYZus{}com\PYZus{}media}\PY{o}{.}\PY{n}{join}\PY{p}{(}\PY{n}{notas\PYZus{}aggr\PYZus{}qtd}\PY{p}{,} \PY{n}{on}\PY{o}{=}\PY{l+s+s2}{\PYZdq{}}\PY{l+s+s2}{filmeId}\PY{l+s+s2}{\PYZdq{}}\PY{p}{)}
         \PY{n}{filmes\PYZus{}com\PYZus{}qtd}\PY{o}{.}\PY{n}{rename}\PY{p}{(}\PY{n}{columns}\PY{o}{=}\PY{p}{\PYZob{}}\PY{l+s+s1}{\PYZsq{}}\PY{l+s+s1}{nota}\PY{l+s+s1}{\PYZsq{}}\PY{p}{:}\PY{l+s+s1}{\PYZsq{}}\PY{l+s+s1}{total\PYZus{}votos}\PY{l+s+s1}{\PYZsq{}}\PY{p}{\PYZcb{}}\PY{p}{,} \PY{n}{inplace}\PY{o}{=}\PY{k+kc}{True}\PY{p}{)}
         
         \PY{n}{filmes\PYZus{}com\PYZus{}qtd}\PY{o}{.}\PY{n}{sort\PYZus{}values}\PY{p}{(}\PY{l+s+s1}{\PYZsq{}}\PY{l+s+s1}{total\PYZus{}votos}\PY{l+s+s1}{\PYZsq{}}\PY{p}{,} \PY{n}{ascending}\PY{o}{=}\PY{k+kc}{False}\PY{p}{)}\PY{o}{.}\PY{n}{head}\PY{p}{(}\PY{l+m+mi}{10}\PY{p}{)}
\end{Verbatim}


\begin{Verbatim}[commandchars=\\\{\}]
{\color{outcolor}Out[{\color{outcolor}40}]:}       filmeId                                      filme  \textbackslash{}
         314       356                        Forrest Gump (1994)   
         277       318           Shawshank Redemption, The (1994)   
         257       296                        Pulp Fiction (1994)   
         510       593           Silence of the Lambs, The (1991)   
         1939     2571                         Matrix, The (1999)   
         224       260  Star Wars: Episode IV - A New Hope (1977)   
         418       480                       Jurassic Park (1993)   
         97        110                          Braveheart (1995)   
         507       589          Terminator 2: Judgment Day (1991)   
         461       527                    Schindler's List (1993)   
         
                                         genero  media\_notas  total\_votos  
         314           Comedy|Drama|Romance|War     4.164134        329.0  
         277                        Crime|Drama     4.429022        317.0  
         257        Comedy|Crime|Drama|Thriller     4.197068        307.0  
         510              Crime|Horror|Thriller     4.161290        279.0  
         1939            Action|Sci-Fi|Thriller     4.192446        278.0  
         224            Action|Adventure|Sci-Fi     4.231076        251.0  
         418   Action|Adventure|Sci-Fi|Thriller     3.750000        238.0  
         97                    Action|Drama|War     4.031646        237.0  
         507                      Action|Sci-Fi     3.970982        224.0  
         461                          Drama|War     4.225000        220.0  
\end{Verbatim}
            
    \hypertarget{desafio-4}{%
\section{Desafio 4}\label{desafio-4}}

Arredondamento dos valores da coluna de notas\_medias

    \begin{Verbatim}[commandchars=\\\{\}]
{\color{incolor}In [{\color{incolor}41}]:} \PY{n}{filmes\PYZus{}com\PYZus{}qtd}\PY{p}{[}\PY{l+s+s1}{\PYZsq{}}\PY{l+s+s1}{media\PYZus{}notas}\PY{l+s+s1}{\PYZsq{}}\PY{p}{]} \PY{o}{=} \PY{n}{filmes\PYZus{}com\PYZus{}qtd}\PY{p}{[}\PY{l+s+s1}{\PYZsq{}}\PY{l+s+s1}{media\PYZus{}notas}\PY{l+s+s1}{\PYZsq{}}\PY{p}{]}\PY{o}{.}\PY{n}{round}\PY{p}{(}\PY{p}{)}
         \PY{n}{filmes\PYZus{}com\PYZus{}qtd}\PY{o}{.}\PY{n}{head}\PY{p}{(}\PY{p}{)}
\end{Verbatim}


\begin{Verbatim}[commandchars=\\\{\}]
{\color{outcolor}Out[{\color{outcolor}41}]:}    filmeId                               filme  \textbackslash{}
         0        1                    Toy Story (1995)   
         1        2                      Jumanji (1995)   
         2        3             Grumpier Old Men (1995)   
         3        4            Waiting to Exhale (1995)   
         4        5  Father of the Bride Part II (1995)   
         
                                                 genero  media\_notas  total\_votos  
         0  Adventure|Animation|Children|Comedy|Fantasy          4.0        215.0  
         1                   Adventure|Children|Fantasy          3.0        110.0  
         2                               Comedy|Romance          3.0         52.0  
         3                         Comedy|Drama|Romance          2.0          7.0  
         4                                       Comedy          3.0         49.0  
\end{Verbatim}
            
    \hypertarget{desafio-5}{%
\section{Desafio 5}\label{desafio-5}}

Descobrir os generos únicos dos filmes

    \begin{Verbatim}[commandchars=\\\{\}]
{\color{incolor}In [{\color{incolor}42}]:} \PY{n}{generos\PYZus{}df} \PY{o}{=} \PY{n}{filmes\PYZus{}com\PYZus{}qtd}\PY{o}{.}\PY{n}{genero}\PY{o}{.}\PY{n}{str}\PY{o}{.}\PY{n}{get\PYZus{}dummies}\PY{p}{(}\PY{l+s+s1}{\PYZsq{}}\PY{l+s+s1}{|}\PY{l+s+s1}{\PYZsq{}}\PY{p}{)}
         \PY{n}{generos} \PY{o}{=} \PY{n}{generos\PYZus{}df}\PY{o}{.}\PY{n}{columns}
         \PY{n}{generos}\PY{o}{.}\PY{n}{tolist}\PY{p}{(}\PY{p}{)}
\end{Verbatim}


\begin{Verbatim}[commandchars=\\\{\}]
{\color{outcolor}Out[{\color{outcolor}42}]:} ['(no genres listed)',
          'Action',
          'Adventure',
          'Animation',
          'Children',
          'Comedy',
          'Crime',
          'Documentary',
          'Drama',
          'Fantasy',
          'Film-Noir',
          'Horror',
          'IMAX',
          'Musical',
          'Mystery',
          'Romance',
          'Sci-Fi',
          'Thriller',
          'War',
          'Western']
\end{Verbatim}
            
    \hypertarget{desafio-6}{%
\section{Desafio 6}\label{desafio-6}}

Contar o número de aparições de cada genero.

    \begin{Verbatim}[commandchars=\\\{\}]
{\color{incolor}In [{\color{incolor}43}]:} \PY{n}{generos\PYZus{}df}\PY{o}{.}\PY{n}{sum}\PY{p}{(}\PY{p}{)}\PY{o}{.}\PY{n}{sort\PYZus{}values}\PY{p}{(}\PY{n}{ascending}\PY{o}{=}\PY{k+kc}{False}\PY{p}{)}
\end{Verbatim}


\begin{Verbatim}[commandchars=\\\{\}]
{\color{outcolor}Out[{\color{outcolor}43}]:} Drama                 4361
         Comedy                3756
         Thriller              1894
         Action                1828
         Romance               1596
         Adventure             1263
         Crime                 1199
         Sci-Fi                 980
         Horror                 978
         Fantasy                779
         Children               664
         Animation              611
         Mystery                573
         Documentary            440
         War                    382
         Musical                334
         Western                167
         IMAX                   158
         Film-Noir               87
         (no genres listed)      34
         dtype: int64
\end{Verbatim}
            
    \hypertarget{desafio-7}{%
\section{Desafio 7}\label{desafio-7}}

Plotar o grafico de aparições por genero. Grafico de barras.

    \begin{Verbatim}[commandchars=\\\{\}]
{\color{incolor}In [{\color{incolor}44}]:} \PY{n}{generos\PYZus{}df}\PY{o}{.}\PY{n}{sum}\PY{p}{(}\PY{p}{)}\PY{o}{.}\PY{n}{sort\PYZus{}values}\PY{p}{(}\PY{n}{ascending}\PY{o}{=}\PY{k+kc}{True}\PY{p}{)}\PY{o}{.}\PY{n}{plot}\PY{p}{(}\PY{n}{kind}\PY{o}{=}\PY{l+s+s1}{\PYZsq{}}\PY{l+s+s1}{barh}\PY{l+s+s1}{\PYZsq{}}\PY{p}{,}\PY{n}{title}\PY{o}{=}\PY{l+s+s1}{\PYZsq{}}\PY{l+s+s1}{Total de filmes por genero}\PY{l+s+s1}{\PYZsq{}}\PY{p}{,}\PY{n}{figsize}\PY{o}{=}\PY{p}{(}\PY{l+m+mi}{16}\PY{p}{,} \PY{l+m+mi}{6}\PY{p}{)}\PY{p}{,}\PY{n}{color}\PY{o}{=}\PY{l+s+s1}{\PYZsq{}}\PY{l+s+s1}{purple}\PY{l+s+s1}{\PYZsq{}}\PY{p}{)}
         \PY{n}{plt}\PY{o}{.}\PY{n}{show}\PY{p}{(}\PY{p}{)}
\end{Verbatim}


    \begin{center}
    \adjustimage{max size={0.9\linewidth}{0.9\paperheight}}{output_87_0.png}
    \end{center}
    { \hspace*{\fill} \\}
    

    % Add a bibliography block to the postdoc
    
    
    
    \end{document}
